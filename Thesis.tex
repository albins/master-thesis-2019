%
% Template for Doctoral Theses at Uppsala
% University. The template is based on
% the layout and typography used for
% dissertations in the Acta Universitatis
% Upsaliensis series
% Ver 5.2 - 2012-08-08
% Latest version available at:
%   http://ub.uu.se/thesistemplate
%
% Support: Wolmar Nyberg Akerstrom
% Thesis Production
% Uppsala University Library
% avhandling@ub.uu.se
%
%%%%%%%%%%%%%%%%%%%%%%%%%%%%%%%%%%%%%%%%%%%


\documentclass{UUThesisTemplate}
%\documentclass[11pt,a4paper,twoside,openany]{report}

\author{Albin Stjerna}
\date{\today}

% Package to determine wether XeTeX is used
\usepackage{ifxetex}

\ifxetex
	% XeTeX specific packages and settings
	% Language, diacritics and hyphenation
        \usepackage[babelshorthands]{polyglossia}
        \usepackage{xunicode}
	\setmainlanguage{english}
	%\setotherlanguages{swedish}

	% Font settings
	\setmainfont{Baskerville}
  \setromanfont{Baskerville}
	\setsansfont{Helvetica Neue}
	\setmonofont{Source Code Pro}
\else
	% Plain LaTeX specific packages and settings
	% Language, diacritics and hyphenation
    % Use English and Swedish languages.
	\usepackage[british]{babel}

	% Font settings
	\usepackage{type1cm}
	\usepackage[latin1]{inputenc}
	\usepackage[T1]{fontenc}
	\usepackage{mathptmx}

	% Enable scaling of images on import
	\usepackage{graphicx}
\fi

\usepackage{listings}
\usepackage{fancyvrb}
\usepackage{multicol}
\usepackage[font={small,it}]{caption}

% Tables
\usepackage{booktabs}
\usepackage{tabularx}
\usepackage{bm}
\usepackage{longtable}
\usepackage{lipsum}

\clubpenalty 4000
\widowpenalty 4000

% Document links and bookmarks
\usepackage{url}
\usepackage[xetex, colorlinks=true,
            linkcolor=blue, citecolor=blue,
            urlcolor=blue,breaklinks]{hyperref}
\usepackage[
    backend=biber,
    natbib=true,
    style=ieee,
    sorting=none,
    backref=true]{biblatex}
\bibliography{../bibliography.bib}

% Numbering of headings down to the subsection level
\numberingdepth{subsection}

% Including headings down to the subsection level in contents
%\contentsdepth{subsection}

\setlength{\columnsep}{0.2cm}
\usepackage{pdfpages}
\usepackage{amsmath}
\usepackage{varioref}
\usepackage{nowidow}
%\usepackage{cleveref}
\usepackage[toc,page]{appendix}
\newcommand{\fixme}[1] {{\color{red}#1}}
\usepackage{microtype}
\usepackage[newfloat]{minted}
\usepackage{csquotes}
\usemintedstyle{xcode}
\setminted{fontsize=\footnotesize}

\setnowidow[5]
\setnoclub[3]

\newcommand{\InRust}[1]{\mintinline{rust}{#1}}

% Uncomment to use a custom abstract dummy text
%\abstractdummy{
%}

\title{A Dataflow Approach to Reference Ownership Analysis for the Rust Programming Language}

\begin{document}
% %\frontmatter*
%     % Creates the front matter (title page(s), abstract, list of papers)
%     % for either a Comprehensive Summary or a Monograph.
%     % Authors of Comprehensive Summaries use this front matter
%     %\frontmatterCS
%     % Monograph authors use this front matter
%     %\frontmatterMonograph

    % Environment used to create a list of papers
    % \begin{listofpapers}
    % 	\item A Paper Discussed in this Thesis \label{apaperlabel}
    % \end{listofpapers}

\maketitle

\section*{Abstract}
\textit{\fixme{This thesis is about something.}}


\begingroup
        % To adjust the indentation in your table of contents, uncomment and enter the widest numbers for each level
        %  E.g.  \settocnumwidth{widest chapter number}{widest section number}{widest subsection number}...{...}
       %  \settocnumwidth{5}{4}{5}{3}{3}{3}
  \tableofcontents
  \listoffigures
  % \listoftables
\endgroup
  
%\section*{Acknowledgements}
%\fixme{I thank everyone for everything.}

\chapter{Introduction}
%\epigraph{\fixme{short quote}}
%\mainmatter{}
% what are the contributions made?
% why?

Rust is a young systems language originally developed at Mozilla
Research~\cite{matsakis_rust_2014}. Its stated intention is to combine
high-level programming language features like automatic memory management and
strong safety guarantees, in particular in the presence of concurrency or
parallellism, with predictable performance and pay-as-you-go abstractions in the
style of C++ and similar systems languages.

One of its core features is the memory ownership model, which enables
compile-time safety guarantees against data races, unsafe pointer dereferencing,
and runtime-free automatic memory management, including for dynamic memory
allocated on the heap. This report describes the implementation of Rust's memory
safety checker, called the borrow checker, in an embedded Datalog engine, as
well as its optimisation.

\section{Background}
Whenever a reference to a resource is created in Rust, its borrowing rules
described in Section~\ref{sec:borrowing-rules} must be respected for as long as
the reference is alive, including across function
calls~\cite{nichols_rust_nodate}. In order to enforce these rules, the Rust
language treats the scope of a reference, called its lifetime, as part of its
type, and also provides facilities for the programmer to name and reason about
them as they would any other type.

Since its release, the Rust compiler has been extended through proposal RFC~2094
to add support for so-called non-lexical lifetimes (NLLs), allowing the compiler
to calculate lifetimes of references based on the control-flow graph rather than
the lexical scopes of variables~\cite{noauthor_rfc_2019}. During the spring of
2018, Nicholas Matsakis began experimenting with a new formulation of the borrow
checker, called Polonius, using rules written in
Datalog~\cite{matsakis_alias-based_2018}. The intention was to use Datalog to
allow for a more advanced analysis while also allowing for better compile-time
performance through the advances done centrally to the fix point solving
provided by the Datalog engine used for the computations~\cite{datafrog}.

Formally, the semantics of Rust's lifetime rules has been captured in the
language Oxide, described by
\citeauthor{weiss_oxide:_2019}~\cite{weiss_oxide:_2019}.

% describe what others have done
% describe what was already there

\section{The Borrowing Rules}\label{sec:borrowing-rules}

\begin{description}
  
\item[Variables must be provably initialised before use] Whenever a variable is
  used, the compiler must be able to tell that it is guaranteed to be
  initialised:
  \begin{minted}{rust}
     let x: u32;
     let y = x + 1; // ERROR: x is not initialised
  \end{minted}
\item[A move deinitialises a variable] Whenever ownership of a variable is
  passed on (a \emph{move} in Rust parlance), e.g. by a method call or
  reassignment, the variable becomes deinitialised:
  \begin{minted}{rust}
    struct Point(u32, u32);
    
    let mut pt = Point(6, 9);
    let x = pt;
    let y = pt; // ERROR: pt was already moved to x
  \end{minted}
\item[There can be any number of shared references] A reference, also called a
  \textit{borrow} of a variable, is created with the \InRust{&} operator, and
  there can be any number of references to a variable:
  \begin{minted}{rust}
    struct Point(u32, u32);
    
    let mut pt = Point(6, 9);
    let x = &pt;
    let y = &pt; // This is fine
  \end{minted}
\item[There can only be one simultaneous live unique reference] Whenever a
  unique reference is created, with \InRust{& mut}, it must be unique:
  \begin{minted}{rust}
    struct Point(u32, u32);
    
    let mut pt = Point(6, 9);
    let x = &mut pt;
    let y = &mut pt; // ERROR: pt is already borrowed
    
    // code that uses x and y
  \end{minted}

  This error happens even if the first borrow is shared, but not if
  either \InRust{x} or \InRust{y} are dead (not used).
  
\item[A reference must not outlive its referent] A reference must go out of
  scope at the very latest at the same time as its referent, which protects
  aganst use-after-frees:
  \begin{minted}{rust}
    struct Point(u32, u32);
    
    let x = {
        let mut pt = Point(6, 9);
        &pt
    };
    
    let z = x.0; // ERROR: pt does not live long enough
  \end{minted}

  In this example, we try to set \InRust{x} to point to the variable \InRust{pt}
  inside of a block that has gone out of scope before \InRust{x} does.
\end{description}

\section{From Lifetimes to Reference Provenance}

\chapter{Investigation}
% \epigraph{\fixme{short quote}}


\section{The Borrow Checker in Datalog}

\section{A Field Study of Borrow Patterns}

\section{Optimising the Borrow Checker}

\chapter{Conclusions and Future Work}
%\epigraph{\fixme{short quote}}

%\begin{appendices}
%\end{appendices}

%\backmatter
\printbibliography[heading=bibintoc]
\end{document}
